\section{Pseudo-random Generators}
\begin{enumerate}[(a)]
	\item Let $G_1, G_2: \bits^\lambda \to \bits^{\lambda + l}$ be two deterministic functions mapping $\lambda$ bits into $\lambda + l$ bits (for $l \ge 1$). You know that at least one of $G_1, G_2$ is a secure PRG, but you don't know which one. Show how to design a secure PRG $G^*:\bits^{2\lambda} \to \bits^{\lambda + l}$ by combining $G_1$ and $G_2$.

	      \begin{solution}
		      I claim that $G^*\deq G_1(s_1) \xor G_2(s_2), s_1, s_2 \ud \bits^\lambda$ is a PRG. Indeed, assume without loss of generality that $G_1$ is a PRG and assume that there exists a PPT attacker $\mathcal{A}$ able to distinguish, with non negligible probability, between $G^*(U_\lambda||U_\lambda)$ and $U_{\lambda + l}$. Then we can build an attacker $\mathcal{A}^*$ that breaks $G_1$ (see figure \ref{cryptoredux:4.a}).
		      \begin{cryptoredux}
			      {4.a}
			      {Base an attack to PRG $G_1$, assuming the existence of a PPT attacker $\mathcal{A}$, able to break the security of $G^*$ with non-negligible probability. $\mathcal{A}^*$ returns a bit $b$ to the challenger $\mathcal{C}$, in order to distinguish.}
			      {}
			      {}
			      {}
			      \receive{\shortstack[l]{
					      $z \casesnew{-stealth}{ G_1(s_1), s_1 \ud \bits^\lambda}{\ud \bits^\lambda}$
				      }}
			      {$z$}{}
			      \cseqdelay
			      \invoke{\shortstack[l]{
					      $s_2 \ud \bits^\lambda$ \\
					      $z^* = z \xor  G_2(s_2)$
				      }}{$z^*$}{}
			      \cseqdelay
			      \return{}{$b$}{}
			      \send{}{$b$}{}
		      \end{cryptoredux}

		      Note that is crucial that $s_1, s_2$ are both uniformly chosen at random, otherwise $G^*$ is not guaranteed to be a secure PRG.
	      \end{solution}

	\item Can you prove that your construction works when using the same seed $s^* \in \bits^\lambda$ for both $G_1$ and $G_2$? Motivate your answer.

	      \begin{solution}
		      A simple counterexample is for $G_2\deq G_1 \xor a$, for some $a \in \bits^{\lambda + l}$. Then $G^*(s^*, s^*) = a, \forall s^*$. Note, however, that if the two seeds are both chosen at random, this happens only with negligible probability, so we can tolerate the existence of such a bad case.
	      \end{solution}
\end{enumerate}