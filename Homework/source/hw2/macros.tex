\usepackage{amsmath}
\usepackage{amssymb}
\usepackage{tikz}
\usepackage{mathtools}
\usepackage{xcolor}
\usepackage{framed}
\usetikzlibrary{tikzmark,decorations.pathreplacing,positioning,shapes,fit,arrows}


% Typesetting shortcuts
\def\xor{\oplus}
\def\and{\wedge}

\def\enc{\texttt{Enc}}
\def\dec{\texttt{Dec}}
\def\kgen{\texttt{KGen}}
\def\tag{\texttt{Tag}}
\def\sign{\texttt{Sign}}
\def\vrfy{\texttt{Vrfy}}
\def\hyb{\texttt{Hyb}}
\def\game{\texttt{Game}}
\def\prover{\texttt{P}}
\def\verifier{\texttt{V}}
\def\transcript{\texttt{Trans}}

\def\P{\mathop{\mathbb P}}

\def\N{{\mathbb N}}
\def\Z{{\mathbb Z}}
\def\G{{\mathbb G}}
\def\QR{{\mathbb{QR}}}

\def\ud{{\leftarrow_\$}} % <--$ uniform distribution
\def\deq{{\coloneqq}}    % := def symbol
\def\bits{{\{0, 1\}}}    % {0, 1} set
\def\modop{\mbox{ mod }} % mod operation
\def\attacker{\mathcal{A}} % A
\def\newattacker{\mathcal{A}^*} % A*
\def\challenger{\mathcal{C}} % C

\newenvironment{solution}{\noindent {\sc \underline{\textbf{SOL}}}}{$\Box$ \medskip} 


% To typeset the header

\def\courseprof{Daniele Venturi}
\def\coursenum{1047622}
\def\coursename{Cryptography}
\def\author{Luigi Russo, 1699981}

\newlength{\tpush}
\setlength{\tpush}{2\headheight}
\addtolength{\tpush}{\headsep}

\newcommand{\handout}[2]{
	\noindent\vspace*{-\tpush}\newline\parbox{\textwidth}{
		\textbf{\coursenum : \coursename} \hfill #1 \newline
		by prof. \courseprof \newline
		\mbox{}\hrulefill\mbox{}}
	\vspace*{1ex}\mbox{}\newline
	\bigskip

	\begin{center}{\Large\bf  #2}\end{center}
		\begin{center}
		solutions by \author
	\end{center}
	\bigskip
}

\newcommand{\homework}[2]{
	\handout{#2}{Homework #1}
	\pagestyle{myheadings}
	\thispagestyle{plain}
	\markboth{#2}{\coursename\space -- homework #1}
}


% To_do shortcut

\newcounter{todo_counter}
\newcommand{\notodo}[1]{
}
\newcommand{\todo}[1]{
	\stepcounter{todo_counter}
	\definecolor{shadecolor}{rgb}{1,0,0} % this is yellow
	\begin{shaded}
		TODO \arabic{todo_counter}: #1
	\end{shaded}
}

%TikZ code "illustrating" the new "brace"
\newcommand{\newbrace}[1][]{
	\begin{tikzpicture}[baseline=-0.5ex]
	\draw[#1] (0.35,0.25) -- (0,0);
	\draw[#1] (0.35, -0.25) -- (0,0);
	\end{tikzpicture}
}

% the optional argument allows you to select the type of arrow 
% you can also customize the "new brace"
\newcommand{\casesnew}[3]%
{\;\newbrace[#1]
	\begin{array}{lcr}
		#2 \\
		#3
	\end{array}
}