\documentclass[twoside]{article}
\usepackage
[pdftex,pagebackref,letterpaper=true,colorlinks=true,pdfpagemode=none,urlcolor=blue,linkcolor=blue,citecolor=blue,pdfstartview=FitH]{hyperref}

\usepackage{amsmath,amsfonts}
\usepackage{graphicx}
\usepackage{enumerate}
\usepackage{amsmath}
\usepackage{amssymb}
\usepackage{mathtools}
\usepackage{todonotes}
\usepackage[secpar = \lambda]{lelu}

\bibliographystyle{alpha}

\setlength{\oddsidemargin}{0pt}
\setlength{\evensidemargin}{0pt}
\setlength{\textwidth}{6.0in}
\setlength{\topmargin}{0in}
\setlength{\textheight}{8.5in}

\setlength{\parindent}{0in}
\setlength{\parskip}{5px}


% Typesetting shortcuts
\def\xor{\oplus}
\def\and{\wedge}
\def\MsgSpace{\mathcal{M}}
\def\cSpace{\mathcal{C}}

\newenvironment{solution}{\noindent {\sc \underline{\textbf{SOLUTION}}}}{$\Box$ \medskip}

% To typeset the header

\def\courseprof{Daniele Venturi}
\def\coursenum{1047622}
\def\coursename{Cryptography}
\def\author{Luigi Russo, 1699981}

\newlength{\tpush}
\setlength{\tpush}{2\headheight}
\addtolength{\tpush}{\headsep}

\newcommand{\handout}[2]{
    \noindent\vspace*{-\tpush}\newline\parbox{\textwidth}{
        \textbf{\coursenum : \coursename} \hfill #1 \newline
        by prof. \courseprof \newline
        \mbox{}\hrulefill\mbox{}}
    \vspace*{1ex}\mbox{}\newline
    \bigskip

    \begin{center}{\Large\bf  #2}\end{center}
    \begin{center}
        solutions by \author
    \end{center}
    \bigskip
}

\newcommand{\exam}[1]{
    \handout{#1}{Exam #1}
    \pagestyle{myheadings}
    \thispagestyle{plain}
    \markboth{#1}{\coursename}
}

\setlength{\marginparwidth}{3cm}

\begin{document}
\exam{May 5, 2020}

\section{A PRG Candidate}

Let $f: \bits \to \bits$ be a one-way permutation, and $G: \bits \to \bits[\secpar + l]$ be a pseudorandom generator with positive stretch (i.e., $l \ge 1$). Analyze the following derived construction of a pseudorandom generator $G'_f : \bits \to \bits[2\secpar + l]$, where $G'(s) := (f(s), G(s))$.

In case you think the derived construction is not secure, exhibit a concrete attack; otherwise, provide a proof of security.

\solution{

    $G'$ is not secure for any possible instantiation. Indeed, assume $G$ be constructed from the OWP $f$, by using as hardcore predicate Goldreich-Levin $l$ times.

    Namely, let $G(s) = h(f(s)) || h(f(f(s))) || \dots || h(f(\dots(f(s)))) || f(\dots(f(s)))$, where $h$ is the Goldreich-Levin hard-core predicate for $f$. In this case we can easily break the security of $G'$, since we are given both $G(s)$ and $f(s)$.
}

\section{PKE Combiners}

Let $\Pi_1 = (\KGen_1, \Enc_1, \Dec_1)$ and $\Pi_2 = (\KGen_2, \Enc_2, \Dec_2)$ be two PKE schemes with the same message space $\MsgSpace = \bits[n]$. You know that at least one of the two PKE schemes is secure, but you don't know which one. Show how to combine $\Pi_1$ and $\Pi_2$ into a PKE scheme $\Pi = (\KGen, \Enc, \Dec)$, with message space $\MsgSpace$, such that $\Pi$ satisfies $\indCPA$ security as long as at least one of $\Pi$ and $\Pi_2$ satisfies $\indCPA$ security.

\solution{

    Let $\Pi = (\KGen, \Enc, \Dec)$ be the following:
    \begin{itemize}
        \item $\KGen$: it returns $(pk, sk)$, with $\pk = (\pk_1, \pk_2)$, $sk = (\sk_1, \sk_2)$, where $(\pk_i, sk_i) \samples \KGen_i$
        \item $\Enc(pk, m)$: returns $(c_1, c_2)$, where $c_1= \Enc_1(pk_1, (m \xor r))$ and $c_2 = \Enc_2(pk_2, r)$ for $r \samples \MsgSpace$.
        \item $\Dec(sk, (c_1, c_2))$: it returns $\Dec_1(sk_1, c_1) \xor \Dec_2(sk_2, c_2)$
    \end{itemize}

    Let assume $\Pi_i$ is $\indCPA$ secure. If $\Pi$ is not, then there exists a valid PPT adversary $\adv$ that we can use as follows to build a PPT attacker $\adv_i$ to $\Pi_i$:

    \begin{itemize}
        \item $\adv_i$ generates $(\pk_{3-i}, \sk_{3-i}) \samples \KGen_{3-i}(\secparam)$
        \item $\adv_i$ receives $(\pk_i)$ from the challenger and forwards to $\adv$ the new public key $\pk = (\pk_1, \pk_2)$
        \item $\adv$ produces the challenge $(m_0, m_1)$. $\adv_i$ samples $r \samples \MsgSpace$ and forwards to the challenger the new pair $(m_0 \xor r, m_1 \xor r)$
        \item the challenger chooses at random a bit $b$ and encrypts $m_b$, thus producing the ciphertext $c_b^*$
        \item $\adv_i$ gives $\adv$ the new ciphertext $(c_1, c_2)$, where $c_i = c_b^*$ and $c_{3-i} \samples \Enc_{3-i}(pk_{3-i}, r)$
        \item $\adv$ returns a bit $b'$, which is passed to the challenger
    \end{itemize}
}

\section{ID scheme based on RSA}

Consider the following ID scheme $\Pi = (\KGen, \Prover, \Verifier)$.

\begin{itemize}
    \item The key generation algorithm first computes parameters $(N, e, d)$ as in the RSA cryptosystem. In particular, $N = p \cdot q$ for sufficiently large primes $p, q$, and $e \cdot d \equiv 1 \mod (p - 1)(q - 1)$ with $e$ a prime number. Hence, it picks $x \samples \Zset_N^*$, computes $y = x^e \mod N$, and it returns $\pk = (N, e, y)$ and $\sk = x$.
    \item One execution of the ID scheme goes as follows:
          \begin{description}
              \item[(1)] The prover $\Prover$ picks $a \samples \Zset_N^*$, and  sends $\alpha = a^e \mod N$ to $\Verifier$;
              \item[(2)] The verifier $\Verifier$ forwards a random challenge $\beta \samples \Zset_e$ to $\Prover$;
              \item[(3)] The prover $\Prover$ replies with $\gamma = x^\beta \cdot a \mod N$, where $x$ is taken from the secret key and $a$ is the same value sampled in the first round.
          \end{description}
    \item The verifier $\Verifier$ accepts a transcript $\tau = (\alpha, \beta, \gamma)$ if and only if \[\gamma^e \cdot y^{-\beta} = \alpha \mod N.\]
\end{itemize}

Prove that $\Pi$ is a canonical ID scheme satisfying completeness, special soundness and honest-verifier zero knowledge under the RSA assumption.

\solution{
    \todo[inline]{Prove completeness, special soundness and honest-verifier zero knowledge.}
}

\end{document}